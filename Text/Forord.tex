\section{Forord}\label{sec:forord} 
Hei og velkommen som tredjeklassing på prosess. I årene framover vil det vente deg mange morsomme, og noen krevende, emner på Institutt for kjemisk prosessteknologi. Prosessmodellering, eller prossmod, er det første av mange fag som bygger på et fundament av modellering. Ved å mestre prossmod vil de framtidige fagene bli lettere å forstå. Dette kompendiet er en slags kortversjon av faget og har som mål å være et pedagogisk hjelpemiddel til deg som student. Forelesere på IKP har nemlig en tendens til å vinkle et ellers enkelt konsept inn i vanskelige fagtermer og mange matematiske bevis. Her vil vi prøve å bruke så mange assosiasjoner som mulig, gi eksempler som du kan kjenne deg igjen i, og legge vekt på det vi mener er det viktigste å lære seg. En foreleser vil kanskje slå hardt ned på dette og si at dette ikke er korrekt på grunn av ditten og datten, men skitt au, det rette er ikke alltid pedagogisk. Kompendiet bygger på $"$The ABC of Process Modelling$"$ skrevet av Heinz Preisig. Vi presiserer at dette kompendiet ikke er offisielt pensum og gir ingen garanti for å stå i faget TKP4106 Prosessmodellering. I tillegg er det mulig faget forandrer seg med årene så kapitler i kompendiet vil ikke nødvendigvis dekke pensum fra år til år. Prossmod er et modningsfag og krever mengdetrening for forståelse, så det er viktig at du ikke tar fram dette kompendiet 3 dager før eksamen med forventning om å lære deg pensum på kort tid. Ta deg tid, gjør øvinger, diskuter faget med andre og vær kritisk til det vi skriver. Noe av det vi skriver bygger på antakelser som ikke alltid vil være realistiske, og da gjelder det å vite når man skal dykke dypere inn i emnet for en bedre forståelse. Har du spørsmål om kompendiet eller oppdager feil, så ikke nøl med å ta kontakt slik at vi kan samle opp alle endringene og eventuelt gi ut en ny utgave. Lykke til med prossmod og husk: «Hvis det går mer inn i hodet enn ut av det så akkumulerer du kunnskap». 

\setcounter{page}{1}
\clearpage
