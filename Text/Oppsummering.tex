\clearpage
\section{Tabell med de viktigste ligninger}
Under har vi ramset opp de ligningene vi mener du burde huske til eksamen, dette gjelder ikke nødvendigvis ligninger som du har blitt introdusert til i tidligere fag. I tillegg så er det viktig å påpeke at noen ligninger, sånn som shell balance, må du lære deg å utlede siden å pugge dem vil sjeldent treffe riktig. 
\begin{table}[H]
    \centering
    \begin{tabular}{c|c}
        Totalt akkumulering & $\Phi = \int_{t_0}^{t}\dot{\Phi}\,dt $  \\[0.2cm]
         Balanse  & $\dot{\Phi} = \hat{\Phi}_{inn} - \hat{\Phi}_{ut} + \Tilde{\Phi}$ \\[0.2cm]
         Konservering & $\dot{\Phi} = \hat{\Phi}_{inn} - \hat{\Phi}_{ut} $\\[0.2cm]
         System av transportligninger & $\underline{\dot{\Phi}} = \textbf{\doubleunderline{F}}\hspace{0.1cm}\underline{\hat{\Phi}}$ \\[0.2cm]
         Transport av ekstensiv variabel & $\hat{\varphi} = -c\frac{\partial \pi}{\partial \underline{\textbf{r}}}$ \\[0.2cm]
         Lineær transportligning & $\hat{\Phi}_{a|b} = -k_{a|b}(\pi_a-\pi_b)$ \\[0.2cm] 
         Generering av stoffmenge & $\vectil{n}=\,V\mymat{N}^T\underline{\Tilde{\xi}}(\vec{c})$ \\[0.2cm]
         Kjemisk potensial & $\mu_i =\, \mu_{i}^0 + RTln(x_i)$ \\[0.2cm]
         Reaksjonsrate & $\Tilde{\xi} = k^r_r\,g(c)$ \\[0.2cm] 
         Arrhenius ligning & $k^r_r(T) =k^0_r\,e^{\frac{-E_{A}}{RT}}$ \\[0.2cm]
         
         Taylor utvidelse av første orde & $x_{i+1} \approx x_i + \frac{dx_i}{dz}\Delta z$ \\[0.2cm]
    \end{tabular}
    \label{tab:my_label}
\end{table}

\appendix
\section{Refferanse}
Heinz A. Presig, \textit{ABC of modelling}, \url{http://folk.ntnu.no/preisig/?page_id=509}