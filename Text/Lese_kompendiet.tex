\clearpage-
\section{Hvordan lese dette kompendiet}\label{sec:lese}
Dette kompendiet er kort oppsummering av pensum. Vi har valgt å bruke et uformelt språk for å gjøre kompendiet mer lettlest så du lettere skjønner konsepter. TKP4106 er undervist på engelsk (gitt Heinz fortsatt underviser) så studentene vil ofte sitte igjen med engelske fagtermer. Vi har valgt å holde språket på norsk og av den grunn mikser vi inn engelske fagtermer der vi mener det ikke finnes en god norsk oversettelse. Faget er veldig slavisk bygd opp og følger en slags oppskrift på modellering. På samme måte har vi valgt å bygge opp kompendiet med en slags $"$kokebok$"$ på hvordan man må tenke i modellering. Vi starter på overflaten også dykker vi dypere inn i faget for å forklare hvordan man kommer fram til forskjellige ligninger. Dette innebærer at vi bruker tidligere kapitler når vi forklarer nye konsepter, så det er anbefalt å lese kompendiet fra start til slutt når du åpner kompendiet for første gang. Til tider vil du oppdage at kompendiet har mangelfulle forklaringer og da vil vi anbefale å bruke ABC-heftet til Heinz for mer dybde. Det er også anbefalt å friske opp på den lineære algebraen fra matte 3, numerikken fra matte 4N, linearisering fra matte 1, prosessteknikk og termo GK. Til forskjell fra mange andre fag du har hatt tidligere er prossmod et fag hvor alle kapitler er koblet tett med hverandre. Det vil si: \textbf{IKKE PUGG}. Du kommer deg ikke gjennom prossmod ved å pugge deg til svarene siden det kreves forståelse for å bygge modeller. Selvfølgelig kan det hjelpe å pugge noen konsepter før man ser sammenhengen til andre kapitler. Nøkkelordet vårt her er forståelse, og ferdigheten i å trekke røde tråder mellom de forskjellige kapitlene. 