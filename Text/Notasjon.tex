\clearpage
\section{Notasjon}\label{sec:notasjon}

Før vi starter med å forklare konsepter tenker vi det er greit å gi en liten innføring i notasjonen som brukes i faget. Notasjonen kan forandre seg etter som faget utvikler seg, så ikke forvent at all notasjon er inkludert i dette kapittelet. Under har vi tatt en liten oppsummering av den viktigste notasjonsbruken i faget. For full forklaring av notasjon; se bakerst i ABC-heftet.

\begin{itemize}
    \item $\vechat{a}$: Strek under er en vektor
    \item $\mymat{A}$: To streker under er en matrise
    \item $\textbf{A}$: Tykk stor bokstav er også matrise, men sjelden brukt i dette faget.
    \item $V$: Bokstav uten strek og tykkelse er en skalar.
    \item $n$,$q$,$m$ brukes henholdsvis for stoff, volum og masse
    \item $\vecdot{n}$: Dott over $\vec{n}$ er akkumulert av $\vec{n}$ (endring av $\vec{n}$ over tid): $\frac{d(\vec{n})}{dt}$
    \item $\vechat{n}$: Hatt over $\vec{n}$ transport av $\vec{n}$.
    \item $\vechat{n}_{A|B}$: Transport av $\vec{n}$ fra A til B
    \item $\vectil{n}$: Tilde over $\vec{n}$ er generert $\vec{n}$ eller forbrukt $\vec{n}$.  
    \item $\mymat{F}^m$: En matrise som inneholder informasjon om m, i dette tilfellet er det masse og vil representere incidence matrix 
    \item \textbf{$\phi$}: Phi er brukt som ekstensiv variabel, se \cref{sec:ekstensive_intensive}
    \item \textbf{$\varphi$}: Curly phi er lik som $\phi$ men er en ekstensiv variabel som i tillegg avhenger av tid og retning. 
    \item \textbf{$\pi$}: Ikke å forveksle med tallet 3.14. Her brukes det ofte som en effortvariabel, trykk, temperatur eller kjemisk potensial, se \cref{sec:indre_effort} 
    \item := : kolon forran er-lik definerer en sammenheng.
    \item $\frac{dx}{dx}\Big|_i$: Stor strek ved siden siden av indikerer at utrykket gjelder for posisjon $i$.
    \item $\Big(\frac{\partial x}{\partial z}\Big)_{y,w}$: Derivasjon av $x$ m.h.p $z$ hvor variabel $y$ og $w$ holdes konstant.
\end{itemize}

\clearpage