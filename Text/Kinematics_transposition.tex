\clearpage
\section{Reaksjoner og reaksjonskinetikk}\label{sec:kinetics}
Reaksjoner sies å være fundamentet i kjemisk industri. Reaksjoner er interaksjoner mellom stoffer som skaper nye stoffer. I \cref{sec:balanse_enkel} så vi hvordan vi kan sette opp en balanse med \cref{eq:balanse_generell}. I dette kapittelet er vi bare interessert i stoffmengde så vi bytter ut $\Phi$ med $n$. Denne delen viser fram mange likninger så det er bare å holde tunga rett i munnen. Vi skal prøve å gå gjennom dem en etter en. Vi starter med å sette opp den generelle balansen for stoffmengde i systemet:

\begin{equation}
    \label{eq:spacies_balance}
    \vecdot{n} = \vechat{n} + \vectil{n}
\end{equation}

La oss si at systemet vårt ikke har transport inn og ut (e.g. batch reaktor, $\vechat{n}=0$) så all akkumulering i systemet vårt skyldes reaksjoner innad i systemet vårt. Uten å bruke mye tid på forklaring så er genereringsleddet i balansen definert som:

\begin{equation}
    \begin{split}
    \label{eq:generation_species}
    \vecdot{n} =&\,  \vectil{n}\\
    \vectil{n}=&\,V\mymat{N}^T\underline{\Tilde{\xi}}(\vec{c})
    \end{split}
\end{equation}



$\mymat{N}^T$ er den støkiometriske matrisen til reaksjonene, $\underline{\Tilde{\xi}}$ er reaksjonsraten som en funksjon av konsentrasjoner og $V$ er volum. Volum ganges inn i likningen for at $\vectil{n}$ skal ha riktig benevning. Reaksjonraten er en funksjon av konsentrasjon og reaksjonsrate-konstant. 

\textbf{Eksempel}: Stoichimetric species matrix\\
Den støkimetriske matrisen blir laget ved å sette alle reksjonene på hver kolonne og hver reaktant/produkt på hver rad. 
\begin{align*}
    R_{x1}:&\hspace{1cm} \ce{4 NH3 + 5 O2 \leftrightarrow } \,\ce{ 4NO + 6 H2O} \\
    R_{x2}:&\hspace{1cm} \ce{2NO + O2 \leftrightarrow }\, \ce{ 2NO2}\\
\end{align*}
\begin{equation}
    \mymat{N} = 
    \bordermatrix{~ & R_{x1} & R_{x2}   \cr
                  \ce{NH3} & 4 & 0  \cr
                  \ce{O2} & 5 & 1 \cr
                  \ce{NO} & 4 & 2 \cr
                  \ce{H2O} & 6 & 0 \cr
                  \ce{NO2} & 0 & 2
                  }
\end{equation}\\
Husk at vi jobber med den transponerte av $\mymat{N}$ i \cref{eq:generation_species}. 
\subsection{Kjemisk potensial}
Men hvordan vet vi hvilken vei som vil dominere reaksjonen? I \cref{sec:transport_effort} lærte vi om effortvariabler som driver transport. I kjemiske reaksjoner er det kjemisk potensial som bestemmer likevekten mellom reaktantene og produktene:
\begin{equation}
    \mu_i = \Big(
    \frac{\partial U}{\partial n_i}
    \Big)_{S,V,n_{j\neq i}}
\end{equation}
Likevel er det vanlig å bruke konsentrasjon som variabel i reaksjonlikninger siden konsentrasjon er lettere å fysisk måle enn kjemisk potensial. Det viktig å vite at med kjemisk potensial i dette avsnittet er at den favoriserte reaksjonen er den som skaper den laveste endringen i kjemisk potensial $\Delta \mu$. 

En alternativ måte å representere kjemisk potensial er i form av avvik fra ideell løsning. Istedet for å gå gjennom utledning fra energibalanser er denne metoden empirisk og tar utgangspunkt i kjent data. Med andre ord så regnes det kjemisk potensialet fra residualet til den ideelle løsningen. 

\begin{equation}
    \begin{split}
    \mu_i =&\, \mu_{i}^0 +\mu_{i}^{excess}\\
    \mu_i =&\, \mu_{i}^0 + RTln(x_i) 
    \end{split}
\end{equation}


\subsection{Bestemme minimum sett med reaksjoner}\label{sec:set_med_reaksjoner}
La oss si at vi ikke vet reaksjonene, men har fått vite at vi har en menge forskjellige stoffer og ønsker å finne ut hvilke reaksjoner som skjer. 

\textbf{Eksempel}:
Vi har reaktantene:

\begin{equation}
    \label{eq:minimum_species}
   \ce{H2O,CH4,CO,CO2,H2}
\end{equation}
Ved å se på stoffene i \cref{eq:minimum_species} trekker vi fram hvilke atomer de 5 stoffene er bygd opp av. Her har vi 3 forskjellige atomer.

\begin{equation}
    \label{eq:minimum_species_atom}
    \ce{H,O,C}
\end{equation}

Ved bruk av \cref{eq:minimum_species} og \cref{eq:minimum_species_atom} konstruerer vi en matrise som viser hvor mange atomer hvert stoff består av.

\begin{equation}
    \mymat{C} = 
    \bordermatrix{~ & \ce{H2O} & \ce{CH4} & \ce{CO} &\ce{CO2} & \ce{H2}  \cr
                  \ce{H} & 2 & 4 & 0 & 0 & 2\cr
                  \ce{O} & 1 & 0 & 1 & 2 & 0\cr
                  \ce{C} & 0 & 1 & 1 & 1 & 0\cr
                  }
\end{equation}
For å finne antall reaksjoner må vi finne nullspacet til matrisen (null$(\mymat{C}$)). Vi ønsker å finne ut hvor mange av radene i $\mymat{C}$ som er lineært avhengige (antall kolonner minus ranken til $\mymat{C}$). Metodikken er å sette matrisen lik null og radredusere (Gauss reduction, løse matrisen) slik vi gjorde i matte 3. Fordi vi er late og regner med at din lineære algebra er på topp gidder vi ikke å vise alle operasjonene (se ABC-heftet kapittel 8.2 hvis du er usikker):

\begin{align}
    \mymat{C} =& 
    \begin{bmatrix}
        3 & 0 & 0 & 2 & 1 \\
        0 & 3 & 0 & -1 & 1 \\
        0 & 0 & 3 & 4 & -1
    \end{bmatrix}
    =
    \begin{bmatrix}
        1 & 0 & 0 & 2/3 & 1/3 \\
        0 & 1 & 0 & -1/3 & 1/3 \\
        0 & 0 & 1 & 4/3 & -1/3
    \end{bmatrix}
    \\[0.5cm]
    \text{null}(\mymat{C}) =& 
    \begin{bmatrix}
        -2 & -1 \\
        1 & -1 \\
        -4 & 1 \\
        3  & 0 \\
        0 & 3
    \end{bmatrix}
     =
     \begin{bmatrix}
        -2/3 & -1/3 \\
        1/3 & -1/3 \\
        -4/3 & 1/3 \\
        1  & 0 \\
        0 & 1
    \end{bmatrix}
\end{align}

Fra nullspacet vårt ser vi at $\mymat{C}$ har 2 uavhengige rader som betyr at $\mymat{C}$ har 2 frie variabler. Med andre ord så er ranken til $\mymat{C}$ 3. Nå som vi vet at vi har to frie variabler kan vi konstruere 2 reaksjoner fra nullspacet vårt:

\begin{equation}
    \mymat{N} = 
    \bordermatrix{~ & R_{x1} & R_{x2}   \cr
                  \ce{H2O} & -2 & -1  \cr
                  \ce{CH4} & 1 & -1 \cr
                  \ce{CO} & -4 & 1 \cr
                  \ce{CO2} & 3 & 0 \cr
                  \ce{H2} & 0 & 3
                  }
\end{equation}

\begin{align*}
    R_{x1}:&\hspace{1cm} \ce{2H2O + 4CO \leftrightarrow CH4 + 3CO2} \\
    R_{x2}:&\hspace{1cm} \ce{H2O + CH4 \leftrightarrow CO + 3H2}
\end{align*}


\subsection{Reaksjonsrater}
\begin{equation}
    \label{eq:reaksjonsrate}
    \underline{\Tilde{\xi}}(\vec{c}) := \mymat{K}\,g_r(\vec{c})
\end{equation}

$\mymat{K}$ er en diagonal matrise bestående av reaksjonratene ($\xi$)\footnote{I RekTek brukes notasjonen $r_i$ for reaksjonsrater.}, hvor hver rad representerer en reaksjon. $g_r(\vec{c})$ er en funksjon av reaktantene til reaksjonen. For å forklare \cref{eq:reaksjonsrate} så disker vi opp med et eksempel. 

\textbf{Eksempel}: Tre reaksjoner\\
Si at vi har tre reaksjoner med tre reaksjonsrater:
\begin{center}
    \ce{A ->[\Tilde{\xi}_1] B}\\
    \ce{B ->[\Tilde{\xi}_2] C}\\
    \ce{C ->[\Tilde{\xi}_3] D}
\end{center}

Dette gir oss en $\underline{\Tilde{\xi}}$ vektor med tre elementer og en 3x3 matrise for $\mymat{K}$. Fra reaksjonslikningen antar vi at reaksjonen bare går én vei, så reaksjonsraten er kun avhengig av konsentrasjonen til reaktanten. Vi setter opp de tre reaksjonlikningene for A,B og C:

\begin{align}
    \Tilde{\xi}_1 =& k^r_1\,c_A \\[0.3em]
    \Tilde{\xi}_2 =& k^r_1\,c_B \\[0.3em]
    \Tilde{\xi}_3 =& k^r_1\,c_C \\
\end{align}

Med tre utrykk for reaksjonratene setter vi dem sammen til et system av likninger:
\begin{align}
    \label{eq:reaksjonsrate_utvidet1}
    \underline{\Tilde{\xi}}(\vec{c}) =& \mymat{K}\,\mymat{S}\vec{c}
    \\
    \label{eq:reaksjonsrate_utvidet2}
    \begin{bmatrix}
    \Tilde{\xi}_1 \\[0.3em]
    \Tilde{\xi}_2 \\[0.3em]
    \Tilde{\xi}_3
    \end{bmatrix}   
    =
    \begin{bmatrix}
    k^r_1 & 0 & 0 \\[0.3em]
    0 & k^r_2 & 0\\[0.3em]
    0 & 0 & k^r_3
    \end{bmatrix}
    \,&
   \begin{bmatrix}
   1 & 0 & 0 & 0\\[0.3em]
   0 & 1 & 0 & 0\\[0.3em]
   0 & 0 & 1 & 0\\[0.3em]
   \end{bmatrix}
   \,
   \begin{bmatrix}
   c_A\\[0.3em]
   c_B\\[0.3em]
   c_C\\[0.3em]
   c_D
   \end{bmatrix}
\end{align}
\cref{eq:reaksjonsrate_utvidet1} er det samme som \cref{eq:reaksjonsrate} hvorav vi har satt inn $\mymat{S}\,\vec{c}$ for $g_r(\vec{c})$. Fra \cref{eq:reaksjonsrate_utvidet2} kan vi se at vi har et uttrykk for hver reaksjonrate. 



\subsection{Reaksjonsratekonstanten}
Reaksjonsratekonstansten er ikke konstant, ironisk nok, men sterkt avhengig av temperatur. Ofte bruker man Arrhenius likning for å beregne denne ``konstanten''. Arrhenius likning kjenner du sikkert igjen fra tidligere fag.

\begin{equation}
    \label{eq:arrhenius}
    k^r_r(T) :=k^0_r\,e^{\frac{-E_{A}}{RT}}
\end{equation}

\begin{align*}
    k^0_r =&\, \text{Pre-exponential factor} \\
    E_A =&\, \text{Aktiveringsenergi} \\
    R =&\, \text{Universell gasskonstant} \\
    T =&\, \text{Temperatur}
\end{align*}


